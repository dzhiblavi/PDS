\section{Транзакции в распределённых системах. ACID. 2 Phase Commit.}

\textit{Про транзакции написано в предыдущем билете.}

\begin{definition} \textit{2 Phase Commit:}
    \begin{itemize}
        \item Централизованный алгоритм завершения транзакции, т.е.
            у каждой транзакции есть выделенный \textit{transaction coordinator}.
        \item \textbf{Фаза 1.} Запрос (request):
            \begin{itemize}
                \item Координатор спрашивает каждого участника о готовности к 
                    завершения транзакции.
                \item Участник может ответить \textit{yes} только, если он может 
                    обеспечить завершение даже в случае сбоя (т.е. он всё записал) 
                    и все данные корректны, иначе \textit{no}.
                \item Транзакцию можно завершить только, если все участники ответили 
                    \textit{yes}.
            \end{itemize}
        \item \textbf{Фаза 2.} Завершение:
            \begin{itemize}
                \item Координатор принимает решение commit/abort и записывает его.
                \item Координатор доводит до участников решение.
            \end{itemize}
    \end{itemize}
\end{definition}

\begin{remark} \textit{Ошибки:}
    \begin{itemize}
        \item Transaction Commit = Consensus, поэтому к нему применим результат 
            \textit{FLP}.
        \item При отказе узлов или связи \textit{2PC} не сможет завершиться, до 
            восстановления узлов/связи.
    \end{itemize}
\end{remark}

\begin{remark}
    Много полезных картинок в конце презентации к лекции 7.
\end{remark}

