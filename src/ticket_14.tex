\section{Диффундирующие вычисления, пример. Останов. Алгоритм Дейксты и Шолтена.}

\begin{definition}
    \textit{Диффундирующим} называется вычисление, для которого верно:
    \begin{itemize}
        \item Процессы бывают в двух состояниях: активный и пассивный.
        \item Получение сообщения делает процесс активным.
        \item Посылать сообщения могут только активные процессы.
        \item Активный процесс в любой момент может стать пассивным.
        \item Алгоритм начинается с одного активного процесса-инициатора.
    \end{itemize}
\end{definition}

\begin{example}
    Алгоритм Дейкстры -- пример диффундирующего вычисления.
\end{example}

\begin{definition}
    Диффундирующее вычисление завершилось, если все процессы пассивны и нет
    сообщений в пути.
\end{definition}

\begin{definition}
    \textit{Проблема останова} -- как процессу-инициатору узнать, когда
    алгоритм завершился?
\end{definition}

\begin{algorithm}(Дейкстра, Шолтен. Останов диффундирующего вычисления)
    \enewline
    \begin{itemize}
        \item Все процессы будут выстраиваться в дерево.
        \item На все сообщения требуются подтверждения.
        \item Каждый процесс знает своего предка в дереве, число своих детей
            и разницу между числом отправленных сообщений, и сообщений,
            на которые было получено подтверждение.
        \item \textit{Зеленым} назовем пассивный процесс без детей
            и неподтвержденных сообщений. В противном случае, процесс
            считается красным. Дерево состоит из красных процессов.
        \item При получении сообщения, зеленый процесс становится красным,
            делая родителем отправителя сообщения и высылая тому подтверждение.
            После получения подтверждения отправитель увеличивает счетчик детей.
        \item Аналогично, как только процесс становится зеленым, он удаляет себя
            из дерева, посылая предку соответствующее сообщение.
        \item Вычисление остановилось, как только корень дерева (то есть, инициатор),
            становится зеленым.
    \end{itemize}
\end{algorithm}

