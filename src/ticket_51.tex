\section{Resilient Distributed Datasets. Мотивация, реализация,
  секционирование датасетов, материализация датасетов.}

\textbf{Мотивация.}
\begin{itemize}
  \item В итеративных алгоритмах удобно производить вычисления,
    последовательно выполняя \texttt{map}, \texttt{filter} и \texttt{flatMap}.
    Однако, каждая операция требует работу с диском для сохранения
    промежуточных данных, которые не нужны в конечном результате.
\end{itemize}

\lstset{
    language=Python,
    backgroundcolor=\color{black!5}, % set backgroundcolor
    basicstyle=\footnotesize,% basic font setting
}

\begin{example}(Ленивые вычисления)

    Хотим посчитать:
    \begin{lstlisting}
result = datasource.fetch()
                   .map(...)
                   .filter(...)
                   .flatMap(...)
                   .collect(...)
    \end{lstlisting}

    Обозначим \texttt{f} --- \texttt{map}, \texttt{p} --- \texttt{filter}, \texttt{g} --- \texttt{flatMap}.

    \textbf{Плохая реализация:}

    \begin{lstlisting}
data = datasource.fetch()

mapped = []
for elem in data:
    mapped.add(f(elem))

filtered = []
for elem in mapped:
    filtered.add(elem) if p(elem)

result = []
for elem in filtered:
    result.addAll(g(elem))
    \end{lstlisting}

    Если считать результат на каждом шаге, то на диске аллоцируется много ненужных коллекций (\texttt{mapped, filtered}).

    \textbf{Оптимальная реализация:}

    \begin{lstlisting}
result = []

for elem in datasources.fetch():
    mapped = f(elem)
    if p(mapped):
        result.addAll(g(mapped))
  \end{lstlisting}

    Аллоцировали на диске только одну коллекцию --- \texttt{result}.
\end{example}

\begin{algorithm}(Последовательная реализация)
  \begin{itemize}
    \item Результат каждой операции --- ленивый контейнер.
    \item Каждый контейнер хранит, с помощью какой операции и из каких контейнеров он был получен.
  \end{itemize}
\end{algorithm}

\begin{algorithm}(Распределенная реализация: Resilient Distributed Datasets)
  \begin{itemize}
    \item Датасет шардируется по строкам на несколько секций.
    \item Для каждой секции известно, с помощью какой операции и из каких секций родительского датасета она была посчитана.
    \item Если шардирование выполнено не по ключу, то для завершения операции может потребоваться
        пересылать данные по сети (например, для операции \texttt{groupByKey}).
    \item Если шардирование выполнено по ключу, то пересылать данные по сети не требуется.
    \item В случае сбоя, утраченную секцию можно пересчитать, поскольку
      известна операция и секции родителя, к которым ее необходимо применить.
      По возможности, это можно сделать параллельно.
    \item В отдельных случаях датасет нужно в явном виде посчитать и сохранить, то есть \textit{материализовать}:
      \begin{itemize}
        \item если требуется совершить несколько запросов к датасету,
        \item если из датасета будет построено более одного производных,
        \item если требуется создать контрольную точку для ускорения
          восстановления после сбоя,
        \item если операция этого требует (например, сортировка).
      \end{itemize}
      Материализацию можно производить как на диск, так и в ОЗУ, в зависимости
      от целей пользователя. Также можно материализовать отдельные секции.
  \end{itemize}
\end{algorithm}
