\section{Multi-Leader репликация. Использование деревьев Мёркла для синхронизации реплик, Sloppy Quorum, Hinted Handoff.}

\begin{algorithm}(Деревья Мёркла)
\begin{itemize}
    \item Построение дерева Меркла: в листьях значение $h(key_i ~\mid~ value_i)$,
        в остальных вершинах хеш от конкатенации значений в вершинах-детях.
        Значение в корне называется \textit{digest}.
    \item Реплики сравнивают \textit{digest}, чтобы узнать о наличии расхождений.
    \item Рекурсивно спускаясь по уровням дерева, реплики найдут различия
        в листовых вершинах. Причем им не нужно пересылать поддеревья,
        в которых нет отличий.
\end{itemize}
\end{algorithm}

\begin{algorithm}(Sloppy Quorum)
  \begin{itemize}
    \item Если читаем с единственной реплики, велик шанс прочесть устаревшее значение
        (например, она может быть медленная).
    \item Будем читать с $R$ реплик и брать самое свежее среди них значение,
        так уменьшается вероятность прочитать устаревшее значение.
        Если прочитали несколько параллельных версий, то решаем конфликт.
    \item Если пишем на одну реплику, то она может быть медленнее,
        рискуем потерять \textit{Durability} при ее падении.
    \item При записи используем $W$ реплик, так запись быстрее отреплицируется.
        Клиенту сообщаем, что запись удалась только после записи на $W$ реплик.
        Также можем делать эти $W$ записей транзакционно.
    \item Пусть $N$ -- количество узлов в кластере, тогда: \\
        $R + W > N$ -- каждая подтверждённая запись будет прочитана \\
        $R + W <= N$ -- улучшается \textit{Durability} записи, уменьшается
        вероятность прочитать старое значение, но можем не прочесть сделанную запись.
  \end{itemize}
\end{algorithm}

\begin{algorithm}(Hinted Handoff)
    Иногда, из-за недоступности, мы не можем сделать запись на $W$ узлов,
    тогда сделаем запись на узлы, которые не должны хранить наши данные
    (например, узлы, являющиеся репликами другой партиции).
    Эти узлы знают, что они хранят записи не принадлежащие им,
    и они отреплицируют эти данные на нужные узлы после их поднятия.
\end{algorithm}
