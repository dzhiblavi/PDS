\section{Paxos. Общие принципы (RSM, концепции). Основные модификации 
(fast paxos и multi paxos).}

\textit{Первую часть вопроса см. в предыдущем билете.}

\begin{definition}
    \textit{Replicated State Machine}. Пусть есть некоторое состояние, которое нужно
    хранить и менять, и которое нужно защитить от сбоя узла. Для надежности можно
    держать несколько копий этого состояния на разных машинах.
    \begin{itemize}
        \item Если операции не коммутируют между собой, то независимо применять
            изменения на разных узлах нельзя.
        \item Поэтому, нужно приходить к консенсусу по вопросу упорядочивания
            операций. В том числе для этого используется Paxos.
    \end{itemize}
\end{definition}

\begin{remark}(Модификации Paxos)
    
    \begin{itemize}
        \item \textbf{Multi Paxos}. Заметим, что лидер может проделать первую фазу
            сразу для нескольких голосований. После этого можно быстро делать
            вторую фазу для всех запусков. Между предлагающим и узнающим
            3 передачи сообщений.
        \item \textbf{Fast Paxos}. Можно еще сильнее сократить задержку.
            Можно предлагать значение сразу принимающим, в случае, если
            предлагающий знает, кто лидер. Получается задержка в 2 сообщения, 
            при отсутствии коллизий. По количеству сообщений может получиться
            хуже, потому что нужно посылать сразу кворуму принимающих.
        \item \textbf{Dynamic Paxos}. Модификация с изменяемым набором серверов.
            Смена списков принимающих становится одной из операций для RSM.
            Основные проблемы находятся на стыке кворумов, нужно чтобы кворумы
            старых и новых процессов были согласны.
        \item \textbf{Cheap Paxos}. Экономим сообщения. Будем посылать не всем
            принимающим, а только $f + 1$ процессу. Остальные будут запасными,
            и использоваться только в случае отказов.
        \item \textbf{Stoppable, Byzantine Paxos}.
    \end{itemize}
\end{remark}
