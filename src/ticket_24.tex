\section{Синхронные системы. Проблема византийских генералов. Алгоритм для N >= 4, f = 1. Объяснить идею обобщения для f > 1.}

Проблема византийских генералов - прийти к консенсусу, штурмовать или не штурмовать крепость, но из N человек, есть f предадетелей.

Решение возможно в синхронной системе только если N > 3f.
1) Все процессы шлют свои предложения.
2) Все процессы пересылают всю полученную информацию всем другим процессам.

Теперь у каждого процесса есть матрица информации от каждого процесса.
Для 4 процессов испорчена в матрице одна строка и столбец. Так как матрица 3 на 3 (без диагонали) в каждой строчке можно определить истинное значение предложение процесса, просто посчитав самое частое значение в строке.
То есть 3 несбойный процесса имеют одни и те же 4 числа и могут прийти к консенсесу.
Предатель не может помешать прийти к консенсусу, но может повлиять на то какое рещение будет принято.
Для f > 1 надо повторить f + 1 фазу (посылаешь предложение, вектор и матрицу предложений, ...)