\section{Общий порядок (total order). Алгоритм Лампорта.}

\textit{Первую часть вопроса см. в билете 16.}

\begin{remark}
    Для случая, когда сообщения отправляются только одному процессу, это
    свойство всегда выполняется.
\end{remark}

\begin{algorithm}(Централизованный алгоритм обеспечения общего порядка)

    Пусть в системе соблюдается FIFO порядок сообщений. Тогда если процесс
    $P$ хочет сделать рассылку сообщения, он сообщает об этом координатору,
    который в свою очередь рассылает сообщения в фиксированном порядке.
\end{algorithm}

\begin{remark}
    Централизованный алгоритм также обеспечивает причинно-согласованный порядок.
\end{remark}

\begin{algorithm}(Лампорт)
    
    Обобщим алгоритм взаимной блокировки. Пусть в системе соблюдается FIFO 
    порядок сообщений. Все multicast-сообщения придется заменить на broadcast.
    Процесс, собирающийся послать сообщения, берет ``билет'', соответствующий
    его логическому времени, и посылает \texttt{request} запрос всем другим
    процессам. Т.е., в свою очередь, отвечают ему \texttt{ok}. После того, как
    был получен \texttt{ok} от всех процессов, отправитель начинает рассылку.
    Как и в алгоритме Лампорта для взаимного исключения, порядок обработки сообщений
    определяется парой из билета и номера процесса.
\end{algorithm}

