\section{Шардирование. Общий принцип, JumpHash.}

\textit{Первую часть вопроса см. в предыдущем билете.}

\begin{algorithm}(JumpHash)
    
    Пусть в системе $N$ узлов.
    \begin{itemize}
        \item Обозначим за
            \[
                0 \leqslant ch(k, N) < N
            \]
            номер узла, на который должен попасть ключ $k$.
        \item При добавлении узла каждый ключ с вероятностью $(N + 1)^{-1}$
            переходит на новый узел.
        \item Все это происходит при предположении, что узлы только добавляются.
            Это разумно в системе, где число данных в основном растет. При отказе
            узла его место (номер) получает новый работающий узел.
    \end{itemize}
\end{algorithm}

\begin{lemma}(О равномерности распределения узлов)

    Пусть $\xi_N = ch(k, N)$ -- случайная величина, номер узла, на котором
    лежит ключ $k$. Тогда $P(\xi_N = i) = \frac{1}{N}$.
\end{lemma}
\begin{proof}
    \enewline
    \begin{itemize}
        \item База индукции. Очевидно, что $P(\xi_1 = 0) = 1$.
        \item Переход. Если $i = N$:
            \[
                P(\xi_{N + 1} = N) = \sum_{i = 0}^{N - 1}{P(\xi_N = i) \cdot 
                \frac{1}{N + 1}} = \sum_{i = 0}^{N - 1}{\frac{1}{N} \cdot 
                \frac{1}{N + 1}} = \frac{1}{N + 1}
            .\]
            Если $i \neq N$:
            \[
                P(\xi_{N + 1} = i) = P(\xi_N = i) \cdot \left(1 - \frac{1}{N + 1}\right)
                = \frac{1}{N} \cdot \frac{N}{N + 1} = \frac{1}{N + 1}
            .\]
    \end{itemize}
\end{proof}

\lstset{
    language=C++,
    backgroundcolor=\color{black!5}, % set backgroundcolor
    basicstyle=\footnotesize,% basic font setting
}

\begin{algorithm}(Наивная реализация)
    Напишем простую реализацию алгоритма, которая для заданного узла
    $k$ эмулирует его жизнь при количестве серверов от $1$ до $n$:
    \begin{lstlisting}
        fun jumpHash(key: Key, n: Int) -> Int {
            random.set_seed(hash(key))
            result = 0
            for (i in 1 until n)
                if (random.uniform(0, 1) < 1/(i + 1))
                    result = i
            return result
        }
    \end{lstlisting}
\end{algorithm}

\begin{algorithm}(Хорошая реализация)
    
    Заметим, что ``прыжки'' происходят редко, то есть достаточно часто
    \[
        ch(k, j + 1) = ch(k, j)
    .\]
    Будем вычислять только точки прыжков, то есть точки, в которых
    \[
        ch(k, j + 1) = j
    .\]
    Предположим, что мы знаем точку последнего прыжка $b$:
    \[
        ch(k, b + 1) = b
    .\]
    Тогда поставим задачу найти ближайшую справа точку, в которой произойдет
    прыжок:
    \[
        ch(k, j + 1) \neq ch(k, b + 1),~ j \to \min_{j > b}
    .\]
    Эквивалентная этой задача ставится так: найти максимальное $j$, в котором
    еще не произошел прыжок:
    \[
        ch(k, j + 1) = ch(k, b + 1),~ j \to \max_{j > b}
    .\]
\end{algorithm}

\begin{lemma}
    $P(ch(k, n) = ch(k, m)) = \frac{m}{n}$, если $n \geqslant m$.
\end{lemma}
\begin{proof}
    \enewline
    \begin{itemize}
        \item Если $n = m$, то
            \[
                P(ch(k, n) = ch(k, n)) = 1 = \frac{n}{n} = \frac{m}{n}
            .\]
        \item Если $n > m$. Тогда прыжков не должно быть на шагах от
            $m + 1$ до $n$. Вероятность того, что на $m + k$-м шаге не произойдет
            прыжок, равна $1 - \frac{1}{m + k} = \frac{m + k - 1}{m + k}$.
            Получаем вероятность:
            \[
                P(ch(k, n) = ch(k, m)) = \frac{m}{m + 1} \cdot \frac{m + 1}{m + 2}
                \cdot \ldots \cdot \frac{n - 1}{n} = \frac{m}{n}
            .\]
    \end{itemize}
\end{proof}

\begin{algorithm}(Хороший алгоритм, продолжение)
    
    Пусть в точке $i \geqslant b + 1$ еще не произошел прыжок. Тогда понятно,
    что $j \geqslant i$:
    \[
        P(j \geqslant i) = P(ch(k, i) = ch(k, b + 1)) = \frac{b + 1}{i}
    .\]
    Воспользуемся этим равенством, чтобы сделать более эффективный алгоритм.
    Сгенерируем случайное число $r ~ U(0, 1)$. Тогда
    \[
        j \geqslant i \Llra r \leqslant \frac{b + 1}{i}
    ,\]
    что эквивалентно
    \[
        j \geqslant i \Llra i \leqslant \frac{b + 1}{r}
    .\]
    Выберем самую точную нижнюю границу на $j$:
    \[
        j = \max_{i \leqslant \frac{b + 1}{r}}{i} = \left\lfloor \frac{b + 1}{r} 
            \right\rfloor
    .\]
    Это и будет очередная точка прыжка. Напишем код, который симулирует жизнь
    ключа:
    \begin{lstlisting}
        fun jumpHash(key: Key, n: Int) -> Int {
            random.set_seed(hash(key))
            b = -1 // Last jump point
            j = 0 // Next jump point
            while (j < n) {
                b = j
                r = random.uniform(0, 1)
                j = floor((b + 1) / r)
            }
        }
    \end{lstlisting}
\end{algorithm}

\begin{lemma}(О времени работы JumpHash)

    Математическое ожидание времени работы JumpHash составляет $\cO(\log{N})$.
\end{lemma}
\begin{proof}
    Мы совершаем прыжки только вперед, причем каждый узел посещаяется не более
    одного раза.
    \[
        \bE[T(N)] = \sum_{i = 1}^{N - 1}{\bE[\xi_i]} = \sum_{i = 1}^{N - 1}{i^{-1}}
        = \cO(\log{N})
    .\]
\end{proof}

\begin{remark}
    \enewline
    \begin{itemize}
        \item JumpHash использует $\cO(1)$ памяти.
        \item JumpHash очень хорошо распределяет нагрузку.
    \end{itemize}
\end{remark}
