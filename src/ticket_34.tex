\section{Raft. Aлгоритм, его свойства.}

\begin{remark}
    \textit{Paxos} обладает недостатками,
    его трудно реализовать на практике:
    \begin{itemize}
        \item Очень сложен в понимании.
        \item Построен на ``однократном консенсусе''
        \item Проблемы с практической реализацией:
            \begin{itemize}
                \item Нужен multi-Paxos.
                \item Нужен выбор лидера.
                \item Нужен общий журнал.
            \end{itemize}
    \end{itemize}
\end{remark}

\begin{definition} \textit{Дизайн Raft}:
    \begin{itemize}
        \item \textit{Понятность} (минимум состояний и недетерминизма).
        \item Подзадачи:
            \begin{itemize}
                \item \textit{Leader election}.
                \item \textit{Log replication}.
                \item \textit{Safety}.
            \end{itemize}
        \item Гарантии:
            \begin{itemize}
                \item \textit{Election Safety} (не более одного лидера).
                \item \textit{Leader Append-Only} (лидер только добавляет).
                \item \textit{Log Matching} (все записи в журналах совпадают).
                \item \textit{Leader Completes} (committed записи будут у будущих лидеров).
                \item \textit{State Machine Safety} (однозначный выбор операции).
            \end{itemize}
    \end{itemize}
\end{definition}

\begin{algorithm} \textit{Выбор лидера:}
    \begin{itemize}
        \item Весь процесс работы Raft разбит на термы, в начале каждого терма
            происходит выбор лидера. Термы нумеруются последовательными числами.
            Каждый узел помнит максимальный номер. В каждом терме не более
            одного лидера.
        \item \textbf{Состояния} узлов:
            \begin{itemize}
                \item \textit{Leader} --- обрабатывает все запросы.
                \item \textit{Follower} --- все узлы, кроме лидера.
                \item \textit{Candidate} --- узлы претендующие на лидерство
                    (роль существует только на этапе выборов).
            \end{itemize}
        \newpage
        \item \textbf{Переходы} состояний:
            \begin{itemize}
                \item Все \textit{followers} следят за \textit{heartbeats}
                    (регулярные сообщения) от лидера. Если лидер не сообщает
                    о себе определённое время, то узел инициирует новые выборы.
                    Для того, чтобы все узлы одновременно не ломились на выборы
                    используют технику рандомизированных таймаутов.
                \item \textit{Candidate} занимается одой из следующих вещей:
                    \begin{itemize}
                        \item Ожидает большинство голосов за себя.
                        \item Участвует в выборе другого кандидата.
                        \item Ожидает таймаут.
                    \end{itemize}
                \item \textit{Leader} работает, пока жив его терм, т.е. до тех пор
                    пока он не обнаружит терм с большим номером в системе.
            \end{itemize}
    \end{itemize}
\end{algorithm}

\begin{definition}
    \textit{Журнал} представляет из себя последовательность пронумерованных
    ячеек (log index), которые хранят номер терма и операцию.
\end{definition}

\begin{algorithm} \textit{Репликация журнала:}
    \begin{itemize}
        \item \textit{Committed} --- записи, которые подтверждены большинством.
        \item \textit{Leader} регулярно рассылает всем \textit{AppendEntries}:
            \begin{itemize}
                \item \textit{leader id} и пачка записей (\textit{log index, term, data}).
                \item Информация (\textit{log index, term}) предыдущей записи.
                \item \textit{Follower} добавляет запись в журнал только в случае,
                    когда информация о предыдущей записи совпадает.
            \end{itemize}
    \end{itemize}
\end{algorithm}

\begin{remark}
    Возможны следующие расхождения в журналах у \textit{followers}:
    \begin{itemize}
        \item Отсутствие каких-то записей (не успели получить обновления).
        \item Неподтверждённые записи (например, какой-то умерший лидер,
            который не успел зафиксировать записи большинством).
        \item Оба вида расхождения вместе (корректный префикс записей и странный набор в хвосте).
    \end{itemize}
\end{remark}

\begin{algorithm} \textit{Согласование журналов}:
    \begin{itemize}
        \item Храним для каждого \textit{follower} next index, т.е. номер
            записи с которой нужно слать обновления.
        \item Если при получение новой записи, информация о предыдущей не совпадает,
            то \textit{follower} должен выкинуть некорректную запись, уменьшить индекс и
            повторить операцию заново, пока не останется корректный префикс. Затем получить
            хвост. (Можно оптимизировать, но смысла нет из-за редкости ошибок).
    \end{itemize}
\end{algorithm}

\newpage
\begin{definition} \textit{Safety механизм}:
    \begin{itemize}
        \item \textit{Committed} записи в журнале не должны перезаписываться.
        \item \textit{Election restriction} --- не отдаём голос лидеру,
            если наш журнал более \textit{свежий}. Для проверки сначала сравниваем
            term последней записи, в случае равенства смотрим на index.
    \end{itemize}
\end{definition}

\begin{remark}
    Только записи из текущего \textit{term leader} считаются
    \textit{committed} при записи в большинство узлов. (Пример проблемы подробно
    рассказан на 8 лекции).
\end{remark}

\begin{proposition}
    \textit{Raft} гарантирует \textit{safety}.
\end{proposition}
