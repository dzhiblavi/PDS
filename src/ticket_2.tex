\section{Формализм. Векторные часы (свойства и алгоритм)}

\begin{definition}
    \textit{Векторные часы}. Определим функцию $VC \colon \bE \to N^k$
    так, чтобы
    \[
        \forall e, f \in \bE~ e \to f \Llra VC(e) < VC(f)
    .\]
    Сравнение производится покомпонентно.
\end{definition}

\begin{algorithm}(Векторное время)
    \begin{itemize}
        \item Каждый процесс хранит свой вектор-время (размер -- число процессов).
        \item Перед посылкой сообщения процесс увеличивает свою компоненту на единицу.
        \item При приеме сообщение берется покомпонентный максимум:
            \[
                VC \leftarrow \max(VC, VC_r)
            .\]
    \end{itemize}
\end{algorithm} 

Свойства векторного времени:
\begin{itemize}
    \item Векторное время уникально для каждого события.
    \item Векторное время полностью передает отношение произошло-до.
    \item 
        \[
            \forall e, f \in \bE\colon~ \texttt{proc(e)} = P_i,~
            \texttt{proc(f)} = P_j \Lra
            \left(e \to f \Llra \begin{pmatrix}
                VC(e)_i \\
                VC(e)_j
            \end{pmatrix} < \begin{pmatrix}
                VC(f)_i \\ 
                VC(f)_j
            \end{pmatrix}\right)
        .\]
\end{itemize}
