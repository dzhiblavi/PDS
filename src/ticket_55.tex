\section{Самостабилизация: взаимное исключение}

\begin{definition} \textit{Самостабилизация:}
    \begin{itemize}
        \item Легальное состояние остаётся легальным
        \item Из любого состояния попадём в легальное через конечное число шагов
    \end{itemize}
\end{definition}

\begin{example} Существует алгоритм самостабилизации для \textit{взаимного исключения}, где ровно один процесс имеет привилегию.
\end{example}

\begin{definition} \textit{Состояние системы:}
    \begin{itemize}
        \item $N$ машин расположены в кольце. Каждая имеет $K$ состояний $(K \geqslant N)$.
            Если состояние меняется, информация отправляется по часовой стрелке дальше.
        \item Машина имеет привилегию, если:
            \begin{itemize}
                \item Для первой: $S = L$. Состояние первой машины равно состонию машины слева
                    (влево это против часовой стрелки)
                \item Для остальных: $S \neq L$
            \end{itemize}
    \end{itemize}
\end{definition}

\begin{definition} \textit{Правила перехода между состояниями:}
    \begin{itemize}
        \item Для первой: $S \neq L \Lra S \coloneqq (S + 1) \mod K$
        \item Для остальных: $S \neq L \Lra S \coloneqq L$
    \end{itemize}
\end{definition}

\begin{theorem} \textit{Данные алгоритм самостабилизируется:}
    \begin{itemize}
        \item Очевидно, что легальное состояние остаётся легальным
        \item Легальное состояние достигается через конечное число ходов:
            \begin{itemize}
                \item Как минимум одна машина имеет привилегию и может ходить
                \item Первая машина сдвинется через через $\cO(n^2)$
                \item Рано или поздно первая машина будет иметь уникальный S
                    (т.к. $K \geqslant N$ и только первая машина имеет право на изменение состояния)
                \item После этого через $\cO(n^2)$ система стабилизируется
            \end{itemize}
    \end{itemize}
\end{theorem}
